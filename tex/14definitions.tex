This section will contain the required definitions. \todo{ADD THEM!!!}

This thesis follows the definitions from \cite{misof-trivial-graphs}. They
are provided in this section for reference.

To denote the solution produced by an algorithm $A$ for an instance $I$ we
use $A(I)$. The cost of a solution $S$ will be denoted by $C(S)$. An
optimal solution for $I$ will be denoted by $Opt(I)$. We will use $E[X]$
to denote the expected value of a random variable $X$.

\begin{definition}\label{def:competitive-ratio}
    Consider an optimization problem in which the goal is to minimize the
    cost of a solution. An algorithm is $c$-competitive if there is a
    constant $\alpha$ such that for each instance $I$ we have $C(A(I))
    \leq c \cdot C(Opt(I)) + \alpha$.  If $\alpha = 0$, we say that $A$ is
    strictly $c$-competitive. The competitive ratio of $A$ is the smallest
    $c$ such that $A$ is $c$-competitive.
\end{definition}

The previous definition can easily be extended to randomized algorithms.
For each instance $I$ we require $E[C(A(I))] \leq c \cdot C(Opt(I)) +
\alpha$. We say that the expected competitive ratio of $A$ is the smallest
value of $c$ satisfying the above inequality.

\begin{definition}\label{def:online-advice}
    An online algorithm $A$ with advice is defined as follows: The input
    for the algorithm is a sequence $X = (x_1, \dots, x_n)$ and an
    infinite advice string $\phi \in \{0, 1\}^\omega$. The algorithm
    produces an output sequence $Y = (y_1, \dots, y_n)$ with the
    restriction that, for all $i$, $y_i$ is computed only from $x_1,
    \dots, x_i$ and $\phi$. This is denoted by $A^\phi(X) = Y$.
\end{definition}

As stated earlier, the computation of $A$ can be interpreted as a series
of turns, where in the $i$-th turn the algorithm reads $x_i$ and yields
$y_i$ using all the information read so far and possibly some additional
bits from the advice string $\phi$. It is worth noting that the definition
does not restrict the computational power of $A$.

\begin{definition}\label{def:advice-complexity}
    The advice complexity of $A$ is a function $s$ such that $s(n)$ is the
    smallest value such that for each input sequence of size $n$ there is
    an advice string $\phi$ such that the algorithm $A$ examines at most
    the first $s(n)$ bits of $\phi$. The advice complexity of an online
    problem is the smallest advice complexity an online algorithm with
    advice needs to produce an optimal solution (i.e., a solution as good
    as an optimal offline algorithm would produce).
\end{definition}

\begin{definition}\label{def:advice-competitive}
    An online algorithm with advice $A$ is $c$-competitive if there is a
    constant $\alpha$ such that for every $n \in \N$ and for every
    instance $I$ of size at most $n$ there is an advice string $\phi$ such
    that $C(A^\phi(I)) \leq c \cdot C(Opt(I)) + \alpha$.
\end{definition}

The previous two definitions suggest another way to look at algorithms
with advice. They have one similarity with the traditional definition of a
non-deterministic Turing machine: in both cases the definitions allow the
existence of an unlimited number of possible computations for a single
input instance and all except ``the right one'' are disregarded. In the
case of a non-deterministic Turing machine this means that in case there
are multiple possibilities for the next configuration, we always choose
the one that leads to an accepting state. In the case of online algorithms
with advice this means that although there is an infinite number of advice
strings, we always choose the one that leads to the optimal solution (or
the required competitive ratio) with the smallest number of accessed bits.

This observation indicates that there is a direct connection between the
number of required advice bits and the number of non-deterministic
decisions a non-deterministic algorithm would have to make in order to
find the expected solution.
