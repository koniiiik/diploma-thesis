Before we delve into the area of advice complexity, we focus on an
analysis of deterministic online algorithms without advice.

Section 13.5 of \cite{dpa-book} presents a proof that no deterministic
algorithm can guarantee a competitive ratio better than linear in the
number of vertices when restricted to strict competitiveness. We reproduce
the proof below.

\begin{theorem}[\cite{dpa-book}]\label{theorem:dpa-deterministic}
    On a path on $L+1$ vertices, any deterministic online algorithm $A$
    has a competitive ratio of at least $L$. Specifically, there exists
    either an input instance $I_1$ where $C(Opt(I_1)) = L$ and
    $C(A(I_1)) = 1$ or an instance $I_2$ for which $C(Opt(I_2)) = 1$ and
    $C(A(I_2)) = 0$.
\end{theorem}

\begin{proof}
    We prove the theorem using an adversary $Adv$. Consider an algorithm
    $A$. The adversary reveals $L$ and issues as the first query $(0, L)$.
    If $A$ rejects this query, $Adv$ terminates the input instance, which
    leads to the second case in the theorem and it means $A$ is not
    competitive.

    If $A$ admits the first query, $Adv$ follows up with $L$ requests:
    $(0, 1),\allowbreak (1, 2),\allowbreak \dots,\allowbreak (L - 1, L)$.
    Since $A$ has already admitted a request spanning the whole path $P$,
    it cannot admit any of these following requests, while the optimal
    solution is to reject the first request and admit all of the following
    $L$ requests. This leads to the first case and means that the
    competitive ratio of $A$ is at least $L$.
\end{proof}

The proof of theorem \ref{theorem:dpa-deterministic} might appear to rely
on a pathologic edge case made possible by the definition of strict
competitiveness: it leans on the fact that each algorithm that denies the
first request can be made non-competitive and setting the parameter
$\alpha$ from definition \ref{def:competitive-ratio} to a value of only
$1$ would eliminate this.

Indeed, Komm described in \cite{komm-thesis} an algorithm that achieves a
competitive ratio of $\left\lceil\frac{L}{\alpha+1}\right\rceil$, which
seems to indicate that relaxing the definition of competitiveness might
lead to better results. However, he also proved that the competitive ratio
of any deterministic algorithm is at least linear in the number of
requests.

Since we study DPA primarily with respect to the length of the
communication network, we complement this with a lower bound on the
competitive ratio, which is one of our new results.

\begin{theorem}\label{theorem:relaxed-dpa-deterministic}
    Consider an arbitrary value of $\alpha$ in the definition of
    competitiveness. On a path on $L+1$ vertices, any deterministic online
    algorithm has a competitive ratio of at least
    $\frac{\lfloor\sqrt{L}\rfloor}{\alpha+1}$.
\end{theorem}

\begin{proof}
    Let $A$ be a $c$-competitive deterministic online algorithm for DPA,
    let $\alpha$ be a positive constant such that $C(A(I)) \geq
    \frac{C(Opt(I))}{c} - \alpha$. We use an adversary $Adv$ to prove the
    bound.

    Let $k := \lfloor\sqrt{L}\rfloor$. $Adv$ starts by issuing
    non-overlapping requests of length $k$: $(0, k), (k, 2k), \dots$,
    until either $A$ admits a request, or $Adv$ submits the request
    $(k(k-1), k^2)$.

    In the former case, let $(ik, (i+1)k)$ be the first (and only) request
    admitted by $A$. $Adv$ then submits the following $k$ requests and
    terminates the input: $(ik, ik+1), (ik+1, ik+2), \dots, ((i+1)k-1,
    (i+1)k)$. Each of these requests overlaps the single admitted request,
    therefore $A$ has to deny all of them.

    The optimal solution for this instance is to admit the first $i$
    requests of length $k$, deny the $i+1$-th request and admit all of the
    following $k$ requests of length $1$, which means $C(Opt(I)) = i+k$,
    while $C(A(I)) = 1$. Since $A$ is $c$-competitive, the following
    inequalities hold.
    \begin{align*}
        1 &\geq \frac{k+i}{c} - \alpha \\
        c &\geq \frac{k+i}{\alpha + 1} \geq \frac{k}{\alpha + 1} =
        \frac{\lfloor\sqrt{L}\rfloor}{\alpha + 1}
    \end{align*}

    In the latter case, $Adv$ terminates the input after request $(k(k-1),
    k^2)$. The optimal solution of this instance is to admit all $k$
    requests, while $A$ rejects everything, which results in these
    inequalities:
    \begin{align*}
        0 &\geq \frac{k}{c} - \alpha \\
        c &\geq \frac{k}{\alpha} \geq \frac{\lfloor\sqrt{L}\rfloor}{\alpha + 1}
    \end{align*}
\end{proof}

This result indicates that even though relaxing the condition of
competitiveness does make it possible to obtain a better competitive ratio
with a deterministic algorithm, it still leaves a significant gap between
the optimal solution and deterministic online algorithms. Therefore in the
rest of this chapter we will adhere to the strict definition, unless noted
otherwise.
