One of the countless ways to categorize algorithmic problems is into
\emph{offline} and \emph{online problems}. Offline problems are those
where the algorighm can access the whole input instance before yielding
the output.  On the other hand, the instance of an online problem is
revealed to the algorithm in smaller pieces and after each piece a partial
solution has to be produced. This partial solution is cannot be changed
later.

A slightly different way of looking at online algorithms is that the
algorithm waits for an input query, procsses it and outputs an answer to
this query immediately. Then it waits for another queryi and repeats the
process until there is nothing more to do.

Solving a problem online is obviously more difficult than solving the same
instance knowing the whole input at once. For many problems it is even
impossible to compute the optimal partial solutions without the knowledge
of the rest of the input sequence. Therefore we define a \emph{competitive
ratio} of an algorithm, which is the quotient of the cost of the solution
produced by the online algorithm and the cost of the optimal solution. An
optimal solution is one produced by an optimal offline algorithm. Since
the competitive ratio can depend on the input instance, we study the worst
competitive ratio an algorithm achieves.

We consider randomized online algorithms as well. In this case we examine
the expected competitive ratio.

Let us describe a few examples of simple online problems to give a better
idea of what they are about. A very simple online problem is ski rental.
Suppose we are going to take an unknown number of ski trips and we do not
own a pair of skis. Renting a pair of skis for a single trip costs $1$,
buying one costs $s$. The input consists of a sequence of queries ``take a
ski trip'' and after each query an answer is expected that is either
``rent'', ``buy'' or ``use skis already bought''. In \cite{skirental} it
is proved that to minimize the competitive ratio the algorithm needs to
rent for the first $s-1$ rounds and then buy a pair of skis; this way, the
competitive ratio is $\frac{2s-1}{s} \approx 2$.
