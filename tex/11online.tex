One of the countless ways to categorize algorithmic problems is into
\emph{offline} and \emph{online problems}. Offline problems are those
where the algorighm can access the whole input instance before yielding
the output.  On the other hand, the instance of an online problem is
revealed to the algorithm in smaller pieces and after each piece a partial
solution has to be produced. This partial solution is cannot be changed
later.

A slightly different way of looking at online algorithms is that the
algorithm waits for an input query, procsses it and outputs an answer to
this query immediately. Then it waits for another query until there is
nothing more to do.

Solving a problem online is obviously more difficult than solving the same
instance knowing the whole input at once. For many problems it is even
impossible to compute the optimal partial solution without the knowledge
of the rest of the input sequence. Therefore we define a \emph{competitive
ratio}, which is the 
