
\cite{dpa-book} contains a simple proof that no deterministic online
algorithm without advice can guarantee a good competitive ratio, which we
reproduce below.

\begin{theorem}\label{theorem:dpa-deterministic}
    On a path of length $L$, any deterministic online algorithm $A$ has a
    competitive ratio of at least $L - 1$. Specifically, there exists
    either an input instance $I_1$ where $C(Opt(I_1)) = L - 1$ and
    $C(A(I_1)) = 1$ or an instance $I_2$ for which $C(Opt(I_2)) = 1$ and
    $C(A(I_2)) = 0$.
\end{theorem}

\begin{proof}
    We prove the theorem using an adversary. Consider an algorithm $A$.
    The adversary reveals $L$ and issues as the first query $(0, L)$. If
    $A$ rejects this query, the adversary terminates the input instance,
    which leads to the second case in the theorem and it means $A$ is not
    competitive.

    If $A$ admits the first query, the adversary follows up with $L - 1$
    requests: $(0, 1), (1, 2), \dots, (L - 1, L)$. Since $A$ has already
    admitted a request spanning the whole path $P$, it cannot admin any
    of these following requests, while the optimal solution is to reject
    the first request and admit all of the following $L - 1$ requests.
    This leads to the first case and means that the competitive ratio of
    $A$ is at least $L - 1$.
\end{proof}

At this point it is worth noting that we restrict ourselves to strict
competitiveness. In the case of general competitiveness with an arbitrary
value of the slackness constant $\alpha$, \cite{komm-thesis} shows a
greedy algorithm that is $\ceil*{\frac{L - 1}{\alpha + 1}}$-competitive.
We complement this result with a matching lower bound: in section
\todo{reference the section} we show that each deterministic algorithm is
at least $\ceil*{\frac{L - 1}{\alpha + 1}}$-competitive.


