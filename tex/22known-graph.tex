\todo{This section is just a result of massive cut\&pasting from various
other sections, will probably need some housekeeping to make sense.}

This section gives an overview of known results about the problem of
online graph coloring. Obviously, the difficulty of this problem depends
greatly on any assumptions we make on the input instance, e.g.
restrictions on the class of graphs, such as trees, bipartite graphs,
cycles or a relationship between the number of vertices and the number of
edges, or the order in which their vertices are revealed to the online
algorithm. All these assumptions provide the algorithm with additional
information. This means that by comparing the advice required to solve
these special cases to the advice complexity of the general case we can
quantify the amount of information provided by a particular set of
assumptions.

An online graph coloring algorithm works roughly as follows. In each turn
a single vertex of the input graph is revealed to the algorithm and it has
to assign a color to this vertex. More precisely, assuming the vertices of
a graph are ordered in a sequence, in $t$-th turn the algorithm has the
knowledge of the subgraph induced by the first $t$ vertices in this
sequence. That means, all edges are revealed as soon as both of their
ending vertices are known.

The order in which vertices are revealed is referred to as the
\emph{presentation order}. In the most general case, the vertices will
appear in a fully arbitrary order. We can restrict this to a connected
presentation order, which means that in each turn the vertex currently
revealed is connected to at least one vertex revealed previously. This can
be restricted even further to the order in which a depth-first search
(\problem{DFS}) or a breadth-first search (\problem{BFS}) will visit
vertices. Another common presentation order is when the sequence of
vertices is sorted by their degrees.

In this thesis, each time we study a particular online graph coloring
problem, we specify explicitly both the class of graphs and the
presentation order. For instance, \graphcol{bipartite}{connected} denotes
that the problem is restricted to bipartite graphs and their vertices are
revealed in a connected order. As a special case, \graphcol{any}{any}
denotes the most general version of the problem where no assumptions are
made at all.

\begin{definition}\label{def:graph-coloring}
    In \problem{OnlineColoring} the instance is an undirected graph $G =
    (V, E)$ with $V = \{1, 2, \dots, n\}$. This graph is presented to an
    online algorithm in turns: In the $k$-th turn the online algorithm
    receives the graph $G_k = G[\{1, 2, \dots, k\}]$, i.e., a subgraph of
    $G$ induced by the vertex set $\{1, 2, \dots, k\}$.  As its reply, the
    online algorithm must return a positive integer: the color it wants to
    assign to vertex $k$. The goal is to produce an optimal coloring of
    $G$ -- the online algorithm must assign distinct integers to adjacent
    vertices, and the largest integer used must be as small as possible.
\end{definition}

When anlyzing a variant of \problem{OnlineColoring}, we always need to
specify the class of graphs it is restricted to and the presentation
order. We denote this using \graphcol{X}{Y} where \problem{X} is the class
of graphs $G$ will belong to and \problem{Y} is the presentation order.
For the class of graphs we will use its common name (e.g.,
``\problem{bipartite}'', ``\problem{planar}'') with the special class
called ``\problem{any}'' meaning that there is no restriction on $G$ at
all. For the presentation order we will use ``\problem{connected}'',
``\problem{bfs}'', ``\problem{dfs}'' and ``\problem{max-degree}'' with
meanings as discussed earlier and, again, ``\problem{any}'' with the
meaning that the vertices may be presented in a fully arbitrary order.

The value of $n$ is not known to the online algorithm beforehand. The
reason for this is that it would provide the algorithm with additional
information about the input instance which may (and in some cases does)
affect the advice complexity of the problem.

Most of the results provided in this section are discussed in more detail
in \cite{misof-trivial-graphs}.

\todo{Add citations to each individual theorem.}

\subsection{General Graphs}

The following asymptotically tight estimates on the advice complexity of
the most general case of online graph coloring have been established.

\begin{theorem}\label{theorem:general-graphs-upper}
    There is an online algorithm with advice which solves
    \graphcol{any}{any} using $n \lg n - n \lg\lg n + O(n)$ bits of
    advice.
\end{theorem}

The general idea is to encode the position of an optimal coloring in a
lexicographically sorted list of all partitions of the set of vertices on
the advice tape.

\begin{theorem}\label{theorem:general-graphs-lower}
    The advice complexity of \graphcol{any}{bfs} is at least $n \lg n - n
    \lg\lg n + O(n)$.
\end{theorem}

The proof of this theorem uses the idea outlined in section
\ref{section:techniques}. It is possible to create a set of instances that
an online algorithm cannot distinguish based on their prefixes up to a
certain length but that require unique colorings in these prefixes
already.

These results are crucial in order to quantify how much a restriction on
the class of graphs simplifies the coloring problem by means of advice
complexity.

\subsection{Bipartite Graphs}

As a reminder, bipartite graphs are those that can be colored using two
colors.

\todo{Mention the cases when advice is not needed.}

\begin{theorem}\label{theorem:bipartite-connected}
    There is a deterministic online algorithm for
    \graphcol{bipartite}{connected} without advice.
\end{theorem}

\begin{proof}
    The algorithm for an optimal coloring is trivial. For the first vertex
    it picks an arbitrary color and afterwards, for each vertex there is
    at least one neighbor whose color has already been assigned. Therefore
    the algorithm just picks the other color.
\end{proof}

This result shows that for bipartite graphs it does not really make any
sense to analyze any of the connected presentation orders. However, for
presentation orders without any restrictions this class of graphs is still
interesting from the point of view of advice complexity.

\subsection{Paths}

Paths are a subclass of bipartite graphs, therefore it is only interesting
to analyze the most general presentation order.

\begin{theorem}\label{theorem:paths-any}
    The advice complexity of \graphcol{path}{any} is
    $\ceil*{\frac{n}{2}}$.
\end{theorem}

\todo{Why?}
