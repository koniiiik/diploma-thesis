\todo{Let's do this.}

Most of the results provided in this section are discussed in more detail
in \cite{misof-trivial-graphs}.

\subsection{General graphs}

The following asymptotically tight estimates on the advice complexity of
the most general case of online graph coloring have been established.

\begin{theorem}\label{theorem:general-graphs-upper}
    There is an online algorithm with advice which solves
    \graphcol{any}{any} using $n \lg n - n \lg\lg n + O(n)$ bits of
    advice.
\end{theorem}

The general idea is to encode the position of an optimal coloring in a
lexicographically sorted list of all partitions of the set of vertices on
the advice tape.

\begin{theorem}\label{theorem:general-graphs-lower}
    The advice complexity of \graphcol{any}{bfs} is at least $n \lg n - n
    \lg\lg n + O(n)$.
\end{theorem}

The proof of this theorem uses the idea outlined in section
\ref{section:techniques}. It is possible to create a set of instances that
an online algorithm cannot distinguish based on their prefixes up to a
certain length but that require unique colorings in these prefixes
already.

These results are crucial in order to quantify how much a restriction on
the class of graphs simplifies the coloring problem by means of advice
complexity.

\subsection{Bipartite graphs}

As a reminder, bipartite graphs are those that can be colored using two
colors.

\begin{theorem}\label{theorem:bipartite-connected}
    There is a deterministic online algorithm for
    \graphcol{bipartite}{connected} without advice.
\end{theorem}

\begin{proof}
    The algorithm for an optimal coloring is trivial. For the first vertex
    it picks an arbitrary color and afterwards, for each vertex there is
    at least one neighbor whose color has already been assigned. Therefore
    the algorithm just picks the other color.
\end{proof}

This result shows that for bipartite graphs it does not really make any
sense to analyze any of the connected presentation orders. However, for
presentation orders without any restrictions this class of graphs is still
interesting from the point of view of advice complexity.
