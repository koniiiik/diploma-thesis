This chapter focuses on the problem of disjoint path allocation. We start
with a definition of the problem, follow with an overview of known results
and in the final section of this chapter we present some new bounds for
the competitive advice complexity of this problem.

\todo{Some motivational shit about DPA.}

In this chapter we build on the results published in \cite{sofsem2014},
therefore we use the same definition of the problem as in the
aforementioned article.

\begin{definition}
    The \emph{disjoint path allocation problem (DPA)} is the following
    maximization problem on a path $P = (v_1, \dots, v_L)$. First, the
    value of $L$ is revealed. Then $n$ requests of the form $(i_k, j_k)$
    follow, where each such pair denotes the subpath of $P$ from $v_{i_k}$
    to $v_{j_k}$. For each pair an algorithm decides whether to admit or
    deny the request. All admitted requests must be pairwise
    edge-disjoint. The goal is to maximize the number of admitted
    requests.
\end{definition}

This problem can be looked at as a series of call requests where each call
has infinite duration and each edge can accommodate at most one call. Note
that the number of requests is not known in advance, only the length of
the path.
