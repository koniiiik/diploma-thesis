Disjoint path allocation is a well-studied specialization of the more
general problem of call admission in arbitrary networks. In the general
case, a dispatcher needs to decide which calls to admit based on the
topology of the network, capacities of its edges, and the bandwidth and
duration of each call.

In the case of disjoint path allocation, we restrict ourselves to a path
on $L + 1$ vertices where all edges have the same capacity, which is equal
to the bandwidth of each call. In addition, each call has an unlimited
duration.

In this chapter we build on the results published in \cite{sofsem2014},
therefore we use the same definition of the problem as in the
aforementioned article.

\begin{definition}[DPA]\label{def:dpa}
    The \emph{disjoint path allocation problem (DPA)} is the following
    maximization problem on a path $P = (v_0, \dots, v_L)$. First, the
    value of $L$ is revealed. Then $n$ requests of the form $(i_k, j_k)$
    follow, where each such pair denotes the subpath of $P$ from $v_{i_k}$
    to $v_{j_k}$. For each pair an algorithm decides whether to admit or
    deny the request. All admitted requests must be pairwise
    edge-disjoint. The goal is to maximize the number of admitted
    requests.
\end{definition}

This problem can be looked at as a series of call requests where each call
has infinite duration and each edge can accommodate at most one call. Note
that the number of requests is not known in advance, only the length of
the path.
