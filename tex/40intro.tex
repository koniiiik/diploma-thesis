The model of algorithms with advice appears to be a useful tool to give a
quantitative measure on the difficulty of online problems, which is more
fine grained than what competitive analysis alone offers. A new question
arises: Is it possible to use advice complexity to quantify the difficulty
of hard offline problems?

The concept of a Turing machine with advice is nothing new -- it has been
introduced in \cite{karp-advice}. However, the motivation behind the
Karp-Lipton model of advice differs significantly from ours -- their model
is closely tied to Boolean circuits and it requires that given a Turing
machine $A$, for every number $n \in \N$ there be some advice string
$\alpha_n$ such that for every input $x$ of size $n$, $A(x, \alpha_n)$
gives the correct answer.

Our interest lies somewhere else: We want to find out what amount of
additional information about a specific input instance can help an
algorithm -- or a Turing machine -- to perform better in terms of time or
space complexity. In other words, while Karp and Lipton allow only a
single advice string for a given size of input, we allow an advice string
specially tailored to each input instance.

The purpose of this chapter is not to give any conclusive results.
Instead, our intention is to lay out the groundwork for future research in
this area.
