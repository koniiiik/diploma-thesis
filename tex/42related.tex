As we have indicated earlier, a different model of Turing machines with
advice by Karp and Lipton has been around for more than 30 years. This
model has been introduced as a means to provide complexity measures for
languages accepted by Boolean circuits. Using this model, various
complexity classes of languages have been analyzed and included in known
complexity hierarchies, such as $\mathbf{P/poly}$ or $\mathbf{L/poly}$,
with important implications on the polynomial hierarchy.

This area of research is, however, very distant from ours. While the
purpose of the Karp-Lipton model is to analyze complexity classes, the
purpose of our model is to analyze individual problems.

Our research has not revealed any paper focusing on a model equivalent to
the one we introduced. This correlates with the fact that thus far,
research in this area has been very limited.

One particular area in which similar research has been done is
cryptanalysis of the RSA cryptosystem, where Coppersmith has shown that it
is possible to find the factors of $N = PQ$ given the high order $1/4 \log
N$ bits of $P$ \cite{factoring}. His method works by reducing the
factoring problem to the problem of finding a root of a bivariate integer
polynomial. This problem can in turn be solved by constructing a basis of
an appropriate lattice and running a basis reduction algorithm in order to
obtain a basis consisting of short vectors. These short vectors are then
used as candidates for a solution.

In terms of our model of advice complexity, the $1/4 \log N$ bits can be
interpreted as advice corresponding to the input $N$. This translates in
an upper bound on the advice complexity of the factoring problem for
inputs $N$ such that $N$ is a product of two primes: for any input of
length $n$ (i.e. the binary representation of $N$ consists of $n$ bits),
$n/4$ bits of advice are sufficient to obtain a solution in polynomial
time.
