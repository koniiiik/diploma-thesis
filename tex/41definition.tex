There is a wide collection of computational models from which we can
choose one to extend with advice. We pick Turing machines due to their
status of the de-facto standard computational model.

\begin{definition}[Turing Machine with Advice]\label{definition:turing-advice}
    A \emph{Turing machine with advice} is a deterministic Turing machine
    $A$ with alphabet $\Sigma = \{0,1\}$, two read-only tapes and one
    read-write tape such that one read-only tape contains the input word
    and the other one contains an infinite binary advice string. $A$
    accepts an input word $x$ if there exists an advice string $\phi$ such
    that when the input tape contains $x$ and the advice tape contains
    $\phi$, $A$ terminates in an accepting state.

    A \emph{Turing machine with advice and output} has an additional
    write-only output tape. Let $S$ be some subset of $\Sigma^*$ and let
    $f: S \to \Sigma^*$ be some function. Turing machine $A$ computes the
    function $f$ if for each $x \in S$ there is an advice string $\phi$
    such that if the input tape contains $x$ and the advice tape contains
    $\phi$, $A$ writes $f(x)$ to the output tape and terminates in an
    accepting state.
\end{definition}

The same considerations as in the case of online algorithms apply here as
well. We define the model to use an infinite advice tape in order to avoid
giving away any additional information in the form of the length of
advice.

This definition resembles nondeterministic Turing machines and in a sense
it is equivalent -- it is easy to see that one computational step of a
nondeterministic Turing machine $A$ with $k$ possible outcomes can be
simulated by $\ceil{\log{}k}$ computational steps of a Turing machine $B$
with advice. In each step, $B$ reads one bit of advice and after reading
all $\ceil{\log{}k}$ bits, decodes the number $k$ indicating which
decision to take. Conversely, we can easily simulate a Turing machine $B$ with
advice using a nondeterministic Turing machine $A$: $A$ can simply
nondeterministically fill one tape used to emulate the advice tape with a
string of nondeterministically chosen length and then proceed to simulate
$B$ without any further changes. The key difference is, as we will see in
a little while, that the model with advice gives us fine-grained control
over how much information the machine can obtain nondeterministically.

The definition of a Turing machine with output is somewhat peculiar. Since
the result of a computation of a given machine $A$ with a given input $x$
can vary depending on the advice string, it is practically impossible to
determine what function $A$ computes. Instead, we choose the opposite
direction. In our work we concentrate on problems for which we know what
the expected output looks like, i.e. we already have a function $f$ and we
want to find a machine such that it can reach the correct output if it
receives correct advice.

It is easy to modify the definition of a machine with output to make it
better suited for analysis of optimization algorithms. Instead of
requiring that the machine computes the correct output, we can define a
cost function whose value is $\infty$ for invalid outputs (or $-\infty$,
depending on whether we talk about a minimization or maximization problem)
and define the output as the outcome with the minimal (or maximal) cost.

Let us now define the complexity measure which lead to the conception of
this computational model in the first place.

\begin{definition}[Offline Advice Complexity]\label{definition:offline-advice-complexity}
    The \emph{advice complexity} of Turing machine $A$ with advice is a
    function $b: \N \to \N$ such that for every input $x$ such that $|x| =
    n$, $A$ accesses at most $b(n)$ positions on the advice tape.
\end{definition}

This complexity measure is analogous to the measure of space complexity in
nondeterministic Turing machines \cite{nspace}, except we are only
interested in the advice tape; on the work tape, $A$ can use as much space
as it needs to. Also note that while the space complexity of a Turing
machine is usually defined in a way that allows any multiplicative
constant, this is not the case of advice complexity. The reason is that
in the case of traditional space complexity, the linear tape compression
theorem holds \cite{tape-compression}, and even though it is possible to
represent multiple advice symbols in one position by extending the advice
alphabet in a similar way, we are specifically interested in the amount of
advice information in terms of the number of bits.
