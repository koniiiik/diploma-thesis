In the previous section we showed that there are problems which cannot be
solved optimally by a deterministic online algorithm. This means that
having access to the whole of the input sequence can help the algorithm to
provide better partial results. However, sometimes it may not be necessary
to access the whole input sequence in order to compute the optimal
solution; in some cases a significantly smaller amount of information is
required.

That is why a computational model of \emph{online algorithms with advice}
has been introduced in \cite{advice-first}. In this model, the online
algorithm is assisted by an oracle with access to the entire input
sequence. The oracle has unlimited computational power and provides the
online algorithm with information about the input sequence that it
requires. We define the \emph{advice complexity} of an online algorithm as
the minimal number of bits it needs to read from the oracle in order to
solve the problem optimally. The advice complexity of an online problem is
then defined as the lowest advice complexity of online algorithms solving
it.

There have been multiple formal definitions of this model with various
drawbacks. \cite{advice-first} contains a definition in which the online
algorithm has access to a finite binary advice tape. That means, however,
that additional information can be encoded into the length of the advice
tape. In \cite{advice-constant} the authors define a slightly different
model where the online algorithm receives the same amount of information
in each round. This makes it impossible to use a sublinear amount of
advice.

The model used in this thesis has been defined in \cite{advice-infinite};
this model uses an infinite advice tape and we measure the number of bits
the algorithm accesses. The following sequence of events can therefore be
imagined: before we start feeding an online algorithm $A$ with the input,
first we give the entire input instance to an oracle which produces a
binary string $\phi$. This binary string is then written at the beginning
of an infinite advice tape which can be accessed by $A$ throughout the
whole computation.

This model of algorithms with advice suggests a similarity with the model
of randomized algorithms. Common definitions of randomized algorithms use
a tape filled with random characters from a certain alphabet, often simply
with random bits. Our model of algorithms with advice can therefore be
looked at as a special case of randomized algorithms, in which the oracle
fills the random tape with the string which leads to the best outcome.

To demonstrate the power of advice, we show the amount of advice required
to solve the two aforementioned online problems optimally. The ski rental
problem is trivial to solve using a single bit of advice -- this bit tells
the algorithm whether there will be at least $s$ queries. The online
algorithm reads this before answering the first query and it knows
immediately whether to buy a pair of skis or just rent them on each trip.

The paging problem is slightly more complex to solve optimally using
advice. Following the proof in \cite{paging-optimal}, this can be done
using $n$ bits of advice. The oracle calculates one optimal solution to
the input instance and assigns a single bit to each request. This bit
indicates whether the page will be accessed again before it is replaced by
another one in the optimal solution, such pages are called active; if the
page will not be accessed again, it is passive. The online algorithm then
just picks a passive page as the victim on each page fault.

Thus far we only covered the amount of advice required to obtain the
optimal solution using an online algorithm. However, it is also useful to
examine the amount of advice required to achieve a certain competitive
ratio and the tradeoff between these two. In this thesis we will study
this aspect as well.

Another possible area of research is the amount of advice required to
solve a \emph{partially online problem}. This is a special case of an
online problem where only a part of the input instance is served in pieces
and at some point the whole rest of the input is served in a single piece.

Taking the previous notion one step further, it also makes sense to apply
the concept of advice to offline problems. In that case, we no longer
study the competitive ratio. Instead, we can use advice to help an
algorithm achieve better efficiency, mainly in terms of its time
complexity, especially for known hard problems, such as $NP$-complete
problems. This direction of research is explored further in the last
chapter of this thesis.
