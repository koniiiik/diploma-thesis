In the previous section we showed that there are problems which cannot be
solved optimally with an online algorithm. This means that having access
to the whole of the input sequence can help the algorithm to provide
better partial results. However, sometimes it may not be necessary to
access the whole input sequence in order to compute the optimal solution,
in some cases a significantly smaller amount of information is required.

That is why a computational model of \emph{online algorithms with advice}
has been introduced in \cite{advice-first}. In this model, the online
algorithm is assisted by an oracle with access to the whole input
sequence. The oracle has unlimited computational power and provides the
online algorithm with information about the input sequence that it
requires. We define the \emph{advice complexity} of an online algorithm as
the minimal number of bits it needs to read from the oracle in order to
solve the problem optimally. The advice complexity of an online problem is
then defined as the lowest advice complexity of online algorithms solving
it.

There have been multiple formal definitions of this model with various
drawbacks. \cite{advice-first} contains a definition in which the online
algorithm has access to a finite binary advice tape. That means, however,
that additional information can be encoded into the length of the advice
tape. In \cite{advice-constant} the authors define a slightly different
model where the online algorithm receives the same amount of information
in each round. This makes it impossible to use a sublinear amount of
advice. The model used in this thesis has been defined in
\cite{advice-infinite}; this model uses an infinite advice tape and we
measure the amount of bits the algorithm accesses.
