Despite the fact that the computational model of online algorithms with
advice has been only concieved a few years ago it is already possible to
notice the emergence of common techniques to analyze online problems and
find lower and upper bounds for their advice complexity.

One of the most basic approaches to find the lower bound on the advice
complexity of a particular online problem is to find a set of instances
with the following properties:

\begin{enumerate}[(i)]
    \item
    for a given non-negative integer $k$ the prefixes $(x_1^{(i)}, \dots,
    x_k^{(i)})$ of instances $I^{(i)}$ are equal, i.e., for two instances
    $I^{(i)} \not= I^{(j)}$, for each $l$ such that $1 \leq l \leq k$,
    the members $x_l^{(i)}$ and $x_l^{(j)}$ are equal

    \item
    for each pair of instances $I^{(i)} \not= T^{(j)}$ there are no
    optimal solutions $Opt(I^{(i)}) = (y_1^{(i)}, \dots, y_{n_i}^{(i)})$,
    $Opt(I^{(j)}) = (y_1^{(j)}, \dots, y_{n_j}^{(j)})$ such that

    $$(y_1^{(i)}, \dots, y_{k}^{(i)}) = (y_1^{(j)}, \dots, y_{k}^{(j)})$$
\end{enumerate}

In other words, we find a set of instances such that the algorithm can't
possibly distinguish the prefixes of these instances but for each instance
a unique solution needs to be yielded in the prefix already. To achieve
this, the advice string must necessarily be used. If the size of this set
of instances is $m$, at least $\lg m$ advice bits need to be accessed
which gives a lower bound on the advice complexity of the problem.

This technique is used in various proofs in \cite{misof-trivial-graphs}.
These will be discussed in more detail in the following sections.
