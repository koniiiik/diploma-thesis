Despite the fact that the computational model of online algorithms with
advice has been only concieved a few years ago it is already possible to
notice the emergence of common techniques to analyze online problems and
find lower and upper bounds for their advice complexity.

To find an upper bound the most straightforward method is, same as with
other complexity metrics, to find an algorithm which solves the problem
and then determine its advice complexity. Any optimal algorithm cannot
then have any worse advice complexity. While this method is obvious, it is
often the most demonstrative one.

One of the most basic approaches to find the lower bound on the advice
complexity of a particular online problem is to find a set of instances
with the following properties:

\begin{enumerate}[(i)]
    \item
    for a given non-negative integer $k$ the prefixes $(x_1^{(i)}, \dots,
    x_k^{(i)})$ of instances $I^{(i)}$ are equal, i.e., for two instances
    $I^{(i)} \not= I^{(j)}$, for each $l$ such that $1 \leq l \leq k$,
    the members $x_l^{(i)}$ and $x_l^{(j)}$ are equal

    \item
    for each pair of instances $I^{(i)} \not= T^{(j)}$ there are no
    optimal solutions $Opt(I^{(i)}) = (y_1^{(i)}, \dots, y_{n_i}^{(i)})$,
    $Opt(I^{(j)}) = (y_1^{(j)}, \dots, y_{n_j}^{(j)})$ such that

    $$(y_1^{(i)}, \dots, y_{k}^{(i)}) = (y_1^{(j)}, \dots, y_{k}^{(j)})$$
\end{enumerate}

In other words, we find a set of instances such that the algorithm can't
possibly distinguish the prefixes of these instances but for each instance
a unique solution needs to be yielded in the prefix already. To achieve
this, the advice string must necessarily be used. If the size of this set
of instances is $m$, at least $\lg m$ advice bits need to be accessed
which gives a lower bound on the advice complexity of the problem.

This technique is used in various proofs in \cite{misof-trivial-graphs}.
These will be discussed in more detail in the following sections.

In \cite{string-guessing} the authors use reductions to a simpler problem
that is easier to analyze as a method to prove lower bounds. Specifically,
they picked the string guessing problem in two variants.

\begin{definition}[String Guessing with Known History]
    The string guessing problem with known history over an alphabet
    $\Sigma$ of size $q \geq 2$ (denoted as \sgkh{q}) is defined as
    follows. The input instance $I = (n, d_1, \dots, d_n)$ consists of an
    integer $n$ specifying the length of the instance and a sequence of
    $n$ characters, where $d_i \in \Sigma, 1 \leq i \leq n$. Let $A$ be an
    online algorithm that solves \sgkh{q}, then $A(I) = (y_1, \dots, y_n,
    \_)$, where $y_i \in \Sigma$. We define the cost of a solution as the
    Hamming distance between the sequence $(y_1, \dots, y_n)$ and the
    sequence $(d_1, \dots, d_n)$, i.e. the number of wrongly guessed
    characters.
\end{definition}

\begin{definition}[String Guessing with Unknown History]
    The string guessing problem with unknown history over an alphabet
    $\Sigma$ of size $q \geq 2$ (denoted as \sguh{q}) is defined as
    follows. The input instance $I = (n, ?_2, \dots, ?_n, d)$ consists of
    an integer $n$ specifying the length of the instance, $n-1$ queries
    without additional information and a string $d = d_1d_2\dots{}d_n$,
    where $d_i \in \Sigma, 1 \leq i \leq n$. Let $A$ be an online
    algorithm that solves \sguh{q}, then $A(I) = (y_1, \dots, y_n, \_)$,
    where $y_i \in \Sigma$. We define the cost of a solution as the
    Hamming distance between the sequence $(y_1, \dots, y_n)$ and the
    sequence $(d_1, \dots, d_n)$.
\end{definition}

Both \sgkh{q} and \sguh{q} consist of $n + 1$ queries where for the first
$n$ queries the algorithm is expected to guess a single character of the
instance; for the last query no meaningful response is expected, its
purpose is only to reveal the input string to allow an offline algorithm
to guess the whole string correctly. The only difference between the two
variants is that in \sgkh{q} it is revealed whether the algorithm guessed
correctly after each guess and in \sguh{q} this is revealed in the last
turn.

For the sake of simplicity, we may sometimes speak about the input string
$d = d_1d_2\dots{}d_n$ instead of the corresponding input instance $I =
(n, d_1, \dots, d_n)$ in the case of \sgkh{q} or $I = (n, ?_2, \dots, ?_n,
d)$ in the case of \sguh{q}.

We formulate the following simple observation.

\begin{observation}\label{observation:sguh-sgkh-bounds}
    It should be noted that any upper bound on the advice complexity of
    \sguh{q} is also an upper bound on the advice complexity of \sgkh{q}
    -- any algorithm that solves \sguh{q} can be used to solve \sgkh{q} as
    well, simply ignoring the characters provided in each query.
    Similarly, any lower bound for \sgkh{q} is also a lower bound for
    \sguh{q}.
\end{observation}

\begin{theorem}\label{theorem:sgkh-upper}\label{theorem:sguh-upper}
    The advice complexity of \sguh{q} is at most $\ceil{n \lg q}$.
\end{theorem}

\begin{proof}
    We prove this theorem by describing an algorithm $A$ using $\ceil{n
    \lg q}$ bits of advice which solves both \sgkh{q} and \sguh{q}.

    The total number of strings of length $n$ is $q^n$. These can be
    sorted in a lexicographic order in which each instance has a position.
    To encode this position, $\ceil{n \lg q}$ bits are required.

    Therefore, after receiving the number $n$ in the first query, $A$
    reads the position $m$ of the string from the advice string and
    enumerates the first $m$ strings of length $n$ in lexicographic order
    until it finds the correct one. Then it just yields one character from
    the string per query.
\end{proof}

\begin{theorem}\label{theorem:sgkh-lower}
    The advice complexity of \sgkh{q} is at least $\ceil{n \lg q}$.
\end{theorem}

\begin{proof}
    We prove this by contradiction. Suppose there is an algorithm $A$
    which solves \sgkh{q} using $m$ bits of advice, $m < \ceil{n \lg q}$.
    The total number of instances of length $n$ is $q^n$. However, using
    $m$ bits of advice it is possible to only encode $2^m \leq 2^{\ceil{n
    \lg q} - 1} < 2^{n \lg q} = q^n$ different values. Therefore, there
    are two input strings $d, d'$ where the same $m$-bit advice string
    $\phi$ leads to the optimal solution.

    Consider the first position $i$ at which strings $d$ and $d'$ differ,
    i.e., $d_i \not= d'_i$. Since $A$ gives the optimal result for the
    input string $d$, in the $i$-th turn it emits $d_i$. However, since up
    until the $i$-th turn, the input is the same for $d'$ as well and
    since the advice string is also the same, $A$ is in exactly the same
    state in the $i$-th turn when processing $d'$ as it is when processing
    $d$. Therefore, for the input string $d'$, $A$ outputs $d_i$ in the
    $i$-th turn as well. This contradicts the assumption that $A$ provides
    an optimal solution for $d'$.
\end{proof}

The following corollary follows from the previous two theorems and
observation \ref{observation:sguh-sgkh-bounds}.

\begin{corollary}
    The advice complexity of both \sgkh{q} and \sguh{q} is $\ceil{n \lg
    q}$.
\end{corollary}

It is easy to see that no deterministic online algorithm without advice
can guarantee to quess even a single character right for alphabets of size
$q \geq 2$. This can be seen by running the algorithm against an adversary
which builds the string $d$ by picking a character different from the one
a deterministic algorithm emits in each turn.

However, by providing an online algorithm with a constant amount of advice
we can already reach a number of correctly guessed characters linear in
the input length.

\begin{theorem}
    Using $\ceil{\lg q}$ bits of advice it is possible to guess at least
    $\ceil{\frac{n}{q}}$ characters correctly.
\end{theorem}

\begin{proof}
    Using $\ceil{\lg q}$ it is possible to encode a single character from
    $\Sigma$. Therefore, by encoding the character with the most
    occurrences in the input string into the $\ceil{\lg q}$ advice bits
    and then emitting this character in each turn we can ensure that the
    algorithm guesses at least $\ceil{\frac{n}{q}}$ correctly.
\end{proof}

The following lower bounds on the number of advice bits required to
guarantee that an algorithm guesses at least a certain amount of
characters right have been established.

\begin{theorem}\label{theorem:sguh-lower-ratio}\label{theorem:sgkh-lower-ratio}
    To guarante that an online algorithm $A$ guesses at least $\alpha{}n$
    characters right for an instance of \sgkh{q} of length $n$ where
    $\frac{1}{q} \leq \alpha < 1$, $A$ needs to access at least the
    following number of advice bits:
    $$
        \paren*{1 + (1 - \alpha)\log_q\paren*{\frac{1 - \alpha}{q - 1}} +
        \alpha\log_q\alpha} n\lg{}q
        =
        (1 - H_q (1 - \alpha)) n\lg{}q
    $$
\end{theorem}

For a proof of this theorem please refer to \cite{string-guessing}.

According to observation \ref{observation:sguh-sgkh-bounds}, the same
lower bound applies to \sguh{q} as well.

The same paper also establishes the following upper bounds on the amount
of advice required to guarantee a certain rate of success.

\begin{theorem}\label{theorem:sguh-upper-ratio}\label{theorem:sgkh-upper-ratio}
    There is an online algorithm that guesses at least $\alpha{}n$
    characters right for an instance of \sguh{q} of length $n$ where
    $\frac{1}{q} \leq \alpha < 1$, which accesses at most the following
    number of advice bits:
    $$
        \ceil*{(1 - H_q(1 - \alpha)) n \lg q + 3 \lg \frac{n}{2} + \lg(\ln
        q) + \frac{1}{2}}
    $$
\end{theorem}

Again, thanks to observation \ref{observation:sguh-sgkh-bounds}, the upper
bound is also valid for \sgkh{q}. The full proof can be found in
\cite{string-guessing}.

The paper then uses these results to establish lower bounds on the advice
required to attain a certain competitive ratio for the online version of
the maximum clique problem and the online set cover problem.

\todo{Study more analysis techniques.}
