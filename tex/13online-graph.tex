The main focus of this thesis is on the problem of online graph coloring.
Obviously, the difficulty of this problem depends greatly on any
assumptions we make on the input instance, e.g. restrictions on the class
of graphs, e.g. trees, bipartite graphs, cycles or a relationship between
the number of vertices and the number of edges, or the order in which
their vertices are revealed to the online algorithm. All these assumptions
provide the algorithm with additional information. This means that by
comparing the advice required to solve these special cases to the advice
complexity of the general case we can quantify the amount of information
provided by a particular set of assumptions.

An online graph coloring algorithm works roughly as follows. In each turn
a single vertex of the input graph is revealed to the algorithm and it has
to assign a color to this vertex. More precisely, assuming the vertices of
a graph are ordered in a sequence, in $t$-th turn the algorithm has the
knowledge of the subgraph induced by the first $t$ vertices in this
sequence. That means, all edges are revealed as soon as both of their
ending vertices are known.

The order in which vertices are revealed is referred to as the
\emph{presentation order}. In the most general case, the vertices will
appear in a fully arbitrary order. We can restrict this to a connected
presentation order, which means that in each turn the vertex currently
revealed is connected to at least one vertex revealed previously. This can
be restricted even further to the order in which a depth-first search
(\problem{DFS}) or a breadth-first search (\problem{BFS}) will visit
vertices. Another common presentation order is when the sequence of
vertices is sorted by their degrees.

In this thesis, each time we study a particular online graph coloring
problem, we specify explicitly both the class of graphs and the
presentation order. For instance, \graphcol{bipartite}{connected} denotes
that the problem is restricted to bipartite graphs and their vertices are
revealed in a connected order. As a special case, \graphcol{any}{any}
denotes the most general version of the problem where no assumptions are
made at all.
